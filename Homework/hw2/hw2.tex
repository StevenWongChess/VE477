%!TeX spellcheck = en-US
\documentclass{article}
\usepackage{bookmark}
\usepackage{color}
\usepackage{amsmath}
\usepackage{hyperref}
\usepackage{listings}
\usepackage{xcolor}
\usepackage{indentfirst}
\usepackage{graphicx}
\usepackage{amsfonts}
\usepackage{hyperref}
\usepackage[top=2cm, bottom=2cm, left=2cm, right=2cm]{geometry}  
\usepackage{algorithm}  
\usepackage{algorithmicx}  
\usepackage{algpseudocode} 
\usepackage{forest}
 
\renewcommand{\algorithmicrequire}{\textbf{Input:}}  
\renewcommand{\algorithmicensure}{\textbf{Output:}}  

\begin{document}
\noindent

%========================================================================
\noindent\framebox[\linewidth]{\shortstack[c]{
\Large{\textbf{VE 477 Homework 2}}\vspace{1mm}\\
Wang Yichao, ID: 517370910011}}

\begin{itemize}

\item \textbf{Exercise 1.}

1. (a) $\lim \limits_{n\to \infty}\frac{n^{3}-3 n^{2}-n+1}{n^3} = 1$. Q.E.D.

(b) $\frac{\ln{n^2}}{\ln{2^n}} = \frac{2\cdot \ln{n}}{\ln{2}\cdot n}$. We take derivative and get it is decreasing when $n > 10$. So $\frac{\ln{n^2}}{\ln{2^n}} < 1$ when $n > 10$. Thus $n^2 < 2^n$ when $n > 10$. Q.E.D.

(c) $\lim \limits_{n\to \infty} \frac{(n+a)^b}{n^b} = (\lim \limits_{n\to \infty}\frac{n+a}{n})^b = 1^b = 1$. Q.E.D.

2. (a) $f(n)=\mathcal{O}(g(n)$

(b) $f(n) = \Omega(g(n))$

3. (a) can not find

(b) f(n) = n and g(n) = 1

4. increasing

with the help of $\log{n} \le n$, we get $f_{2}(n)<f_{1}(n)<f_{3}(n)<f_{4}(n)$

\item \textbf{Exercise 2.}

1. (a)

\begin{center}
\begin{forest}
[$f(n)$,draw
	[$f(n/b)$,draw
		[$f(n/b^2)$,draw
			[$\cdots$]
			[$\cdots$]
			[$\cdots$]
		]
		[$\cdots$]
		[$f(n/b^2)$,draw
			[$\cdots$]
			[$\cdots$]
			[$\cdots$]
		]
	]
	[$\cdots$]
	[$\cdots$]
	[$\cdots$]
	[$\cdots$]
	[$\cdots$]
	[$f(n/b)$,draw
		[$f(n/b^2)$,draw
			[$\cdots$]
			[$\cdots$]
			[$\cdots$]
		]
		[$\cdots$]
		[$f(n/b^2)$,draw
			[$\cdots$]
			[$\cdots$]
			[$\cdots$]
		]
	]
]
\end{forest}
\end{center}

(b) depth is $\log_b n + 1$, number of leaves is $a^{\log_b n}$, the total cost at depth k is $a^k\cdot f(\frac{n}{b^k})$. Just recursive plug in the formula with n, $\frac{n}{b}$, $\frac{n}{b^2}$ ... We can get $T(n) = a^{\log_{b} n} \cdot T(1) + \sum_{j=0}^{\log _{b} n-1} a^{j} f\left(n / b^{j}\right) = n^{\log_{b} a} \cdot T(1) + \sum_{j=0}^{\log _{b} n-1} a^{j} f\left(n / b^{j}\right) = \Theta\left(n^{\log _{b} a}\right)+\sum_{j=0}^{\log _{b} n-1} a^{j} f\left(n / b^{j}\right)$.

2. (a) (i)since $f(n)= \Theta\left(n^{\log _{b} a}\right) = \Theta\left(a^{\log _{b} n}\right)$, we get $f(\frac{n}{b^j})=\Theta\left(a^{\log _{b} \frac{n}{b^j}}\right)$, thus $a^j\cdot f(\frac{n}{b^j})=\Theta\left(a^j\cdot a^{\log _{b} \frac{n}{b^j}}\right) = \Theta\left(a^j\cdot \frac{n}{b^j}^{\log _{b} a}\right)$. 
Adding them up and we can derive the answer. Q.E.D.

(ii) actually $a^j\cdot a^{\log _{b} \frac{n}{b^j}} = a^j\cdot a^{\log _{b} n - j} =  a^{\log _{b} n} = n^{\log _{b} a}$. Since $\log _{b} n$ terms on LHS, the value of LHS should be $n^{\log _{b} a} \log _{b} n$. Q.E.D.

(iii) Since there is only a constant difference between $\log _{b} n$ and $\log n$, it is obvious that $g(n)=\Theta\left(n^{\log _{b} a} \log n\right)$ with the help of (ii) i just proved.

(b) (i) The proof is exactly the same(only difference is the big O instead of theta)(it is dirty work if we want prove by definition)

(ii) $a^{j}\left(\frac{n}{b^{j}}\right)^{\log _{b} a-\varepsilon} = a^{j}\left(a-\varepsilon \right)^{\log _{b} \frac{n}{b^{j}}} = \frac{a^j}{(a-\varepsilon)^j} \cdot n^{\log _{b} a-\varepsilon}$.

Thus we can get $\sum_{j=0}^{\log _{b}{n} -1} a^{j}\left(\frac{n}{b^{j}}\right)^{\log _{b} a-\varepsilon} = n^{\log _{b} a-\varepsilon}\cdot \sum_{j=0}^{\log _{b} n-1} b^{\varepsilon j} = n^{\log _{b} a-\varepsilon} \frac{n^{\varepsilon}-1}{b^{\varepsilon}-1}$.

(iii) plug in the result of (ii), it is obvious since $n^{\log _{b} a-\varepsilon} = n^{\log _{b} a}/{n^\varepsilon}$, and $\varepsilon > 0$. 

$\lim \limits_{n\to \infty} \frac{n^{\varepsilon}-1}{b^{\varepsilon}-1} n^{\log _{b} a-\varepsilon}/ n^{\log _{b} a} = \frac{1- n^{-\varepsilon}}{b^{\varepsilon}-1} = \frac{1}{b^\varepsilon - 1}$, so $g(n)=\mathcal{O}\left(n^{\log _{b} a}\right)$.

(c) (i) 0when j = 0, we have term f(n), also it is obvious that other terms of $g(n) > 0$. So $g(n)=\Omega(f(n))$.

(ii) it is trivial. just use $a f(n / b) \leq c f(n)$ j times with $n = n, n/b, n/b^2 ... n/b^{j-1}$, since < has transition property.

(iii) from (ii), we have $g(n)\leq \sum_{j=0}^{\log _{b} n-1} c^j\cdot f(n) = f(n)\cdot \frac{1-c^{\log_b n}}{1-c}$, which is the stuff we want to prove. Q.E.D.

(iv) from (i) and (iii), we can get the result.

3. Weak Master Theorem

assume $a\ge 1, b > 1$ two constants. the recurrence relation is $T(n) = aT(\frac{n}{b})+f(n)$. Then we can get the asymptotic bound of T(n) by 

\begin{center}
$T(n)=\left\{\begin{array}{lr}\Theta\left(n^{\log _{b} a}\right) & f(n)=\Theta\left(n^{\log _{b} a}\right) \\ \Theta\left(n^{\log _{b} a}\right) & f(n)=O\left(n^{\log _{b} a-\varepsilon}\right) \\ \Theta(f(n)) & a f(n / b) \leq c f(n)\end{array}\right.$
\end{center}



\item \textbf{Exercise 3.}

\begin{algorithm}[H]  
    \caption{mult}  
    \begin{algorithmic}[1]  
        \Require a positive integer n
        \Ensure all the Ramanujam numbers smaller or equal to n
        \State let list[ ] be an array of n zeros.
        \State let result[ ] be an empty list
        \For{ each interger $ 1\leq i \leq \sqrt[3]{n}$ }
        	\For{ each interger $ 1\leq j \leq i$ }
        		\If {$i^3+j^3 \leq n$}
        			\State list[$i^3+j^3$] ++;
        		\EndIf
        	\EndFor
        \EndFor
    	\For{ each interger $ 1\leq k \leq n$ }
       		\If {$2 \leq list[k]$ }
       			result.append(k)
       		\EndIf
        \EndFor
        \State \Return result[ ]
    \end{algorithmic}  
\end{algorithm}

the iteration with two fors has time complexity of $1^{2}+2^{2}+\cdots+\sqrt[3]{n}^{2}=\mathcal{O}(n)$, for the second part that check duplicate elements, the time complexity is simply $\mathcal{O}(n)$. So the total complexity is still $\mathcal{O}(n)$.


\item \textbf{Exercise 4.}
WLOG, we can assume pirate a b c d e f with increasing age. We simply use backward induction. 6 iterations are needed.

For a, it will propose itself 300. 

For b, since half vote can ensure winning, b will give itself 300.

For c, c should give a 1 gold to get a happy and vote for c. no need to give b money.(notice that c need 300 to make b happy, which is nonsense)

similarly, for d, it will give 1 gold to b. For e, it will give a c 1 gold each. For f, it will give b d 1 gold each.

So the final result should be 6 pirates alive and the coin distribution is $\{0,1,0,1,0,298\}$ from young to old.


\end{itemize}

%========================================================================
\end{document}
















