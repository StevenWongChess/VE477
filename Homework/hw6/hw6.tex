%!TeX spellcheck = en-US
\documentclass{article}
\usepackage{bookmark}
\usepackage{color}
\usepackage{amsmath}
\usepackage{hyperref}
\usepackage{listings}
\usepackage{xcolor}
\usepackage{url}
\usepackage{indentfirst}
\usepackage{graphicx}
\usepackage{amsfonts}
\usepackage{hyperref}
\usepackage[top=2cm, bottom=2cm, left=2cm, right=2cm]{geometry}  
\usepackage{algorithm}  
\usepackage{algorithmicx}  
\usepackage{algpseudocode} 
\usepackage{forest}
\newcommand{\bigO}{\mathcal{O}}
\usepackage{listings}
 
\renewcommand{\algorithmicrequire}{\textbf{Input:}}  
\renewcommand{\algorithmicensure}{\textbf{Output:}}  

\begin{document}
\noindent

%========================================================================
\noindent\framebox[\linewidth]{\shortstack[c]{
\Large{\textbf{VE 477 Homework6}}\vspace{1mm}\\
Wang Yichao, ID: 517370910011}}

\begin{itemize}

\item \textbf{Exercise 1.}
1. By definition, there are $n\!$ monomials, and each monomial corresponds to a permutation. We discuss in two cases. (a) If there is a perfect match, WLOG we can assume the elements on one diagonal line are variables $X_i,j$. All the rest are 0. Then $det(A)$ is the multiple of all these variables which is not zero. (b) If there is not a perfect match, then at least one node is not connected, which leads to a row in the matrix to be 0. This will immediately make $det(A)$ to be zero.

2. 
\begin{algorithm}[H]  
    \caption{Lovasz}  
    \begin{algorithmic}[1]  
        \Require Graph $G = (V, E)$ and its matrix A
        \Ensure a boolean that indicates whether there is a perfect match

        \For {each $X_{i,j}$ in A}
            \State $X_{i,j} = uniformly\ random\ from\ 1\ to\ n^2$
        \EndFor
        \If {$det(A)$ == 0}
            \State \Return False
        \Else
            \State \Return True
        \EndIf
    \end{algorithmic}  
\end{algorithm}

3. Time complexity would be $|V|^2$ since we need to set the random number for the whole matrix. Error probability would be $1-\frac{1}{n}$ according to Schwartz-Zippel Lemma.

4. This strategy can be faster than the deterministic algorithm, because it is easier to calculate a determinant with value than a determinant with variables(it can be an expensive $n^2-$variate polynomial of degree n). And the correct probability is quite high.

\item \textbf{Exercise 2.}
1. 
\begin{algorithm}[H]  
    \caption{middle node}  
    \begin{algorithmic}[1]  
        \Require HeadNode
        \Ensure MiddleNode

        \State MiddleNode = HeadNode
        \State EndNode = HeadNode
        \While {EndNode.next $!=$ NULL}
            \State MiddleNode = MiddleNode.next
            \State EndNode = EndNode.next.next
        \EndWhile
        \Return MiddleNode
        
    \end{algorithmic}  
\end{algorithm}

2.
\begin{algorithm}[H]  
    \caption{cycle}  
    \begin{algorithmic}[1]  
        \Require HeadNode
        \Ensure A boolean indicate whether there is a cycle
        \State slow = HeadNode
        \State fast = HeadNode.next
        \While {fast.next $!=$ NULL}
            \If { slow == fast}
                \State \Return True
            \EndIf
            \State slow = slow.next
            \State fast = fast.next.next
        \EndWhile
        \State \Return False

    \end{algorithmic}  
\end{algorithm}
Since we need to traverse the list, $\bigO (n)$ time complexity, and no extra space needed, so $\bigO (1)$ space complexity.

\item \textbf{Exercise 3.}
1. Since n kinds of coupons, we need at least n boxes.

2. The probability of getting a new couple given that we have already get $j-1$ couples is $\frac{n-j+1}{n}$. Then we calculate the reciprocal and get $E\left[X_{j}\right] = \frac{n}{n-j+1}$.

3. It is super interesting that some clever genius can finish question 3 easily by skipping question 2.

To calculate $E\left[X\right]$, we need to add up $E\left[X_{j}\right]$. $E\left[X\right] = p = \sum_{j=1}^n \frac{n}{n-j+1} = n\times \sum_{j=1}^n \frac{1}{j} = n(\log n + \gamma)$, where $\gamma$ is the Euler constant and n goes to infinity. So we get $E\left[X\right] = \Theta (n\log n)$.

4. The formula above shows that the number of boxes grows a little bit faster than the growth of coupon, which is linear. If the couples are identical and we only need to collect n pieces of coupons, that will be linear. So we can make more profit by introducing n different kinds of coupons.







\end{itemize}

%========================================================================
\end{document}
















