\documentclass[catalog.tex]{subfiles}

% do not write anything in the preamble

\begin{document}

\def\pbname{Problem name} %change this, do not use any number, just the name

\section{\pbname} 

% only for overview, so short description (no more than 1-2 lines)
\begin{overview}
\item [Algorithm:] name~(algo.~\ref{alg:\currfilebase}) 
	% -	must match the label of the algorithm 
	% - when writing more than one algo use alg:\currfilebase_a, alg:\currfilebase_b, etc.
\item [Input:] what inputs are expected
\item [Complexity:] complexity of the algorithm, e.g. $\mathcal{O}(n)$
\item [Data structure compatibility:] data structures that can be used with the algorithm; N/A if unrelated
\item [Common applications:] most common fields where this algorithm is used
\end{overview}


\begin{problem}{\pbname}
	Precise and concise formal definition of the problem. No long paragraph here, only a few lines.
\end{problem}


\subsection*{Description}
Detailed description of the problem; More detailed information on the input and complexity; more applications with details on how they relate to each other (if this is the case). Do not hardcode references,  instead use the {\tt \textbackslash label} and {\tt \textbackslash reference} commands.  Examples: citation~\cite{ve477}, a group of figures (Fig.~\ref{fig:\currfilebase_group}), a sub-figure (Fig.~\ref{fig:\currfilebase_a}). To display a new line skip a line in the source code, do not use {\tt \textbackslash\textbackslash}.


\begin{figure}[!htb]
	\centering
	\subfloat[Pic. 1\label{fig:\currfilebase_a}]{\includegraphics{\currfilebase_a}}
	\hspace{2cm} %\qquad
	\subfloat[Pic. 2\label{fig:\currfilebase_b}]{\includegraphics{\currfilebase_b}}
	\caption{Group of pictures}
	\label{fig:\currfilebase_group}
\end{figure}

% add comment in the pseudocode: \cmt{comment}
% define a function name: \SetKwFunction{shortname}{Name of the function}
% use the defined function: \shortname{$variables$}
% use the keyword ``function'': \Fn{function name}, e.g. \Fn{\shortname{$var$}}
\begin{Algorithm}[Name\label{alg:\currfilebase}]
	% -	must match the reference in the overview
	% - when writing more than one algo use alg:\currfilebase_a, alg:\currfilebase_b, etc.
	%\SetKwFunction{myfunction}{MyFunction}	
	\Input{}
	\Output{}
	%	\Fn{\myfunction{$a,b$}}{
	%	}
	\BlankLine

	\Ret

\end{Algorithm}


% include references where to find information on the given problem using latex bibliography
% insert references in the text (\cite{}) and write bibliography file in problem-nb.bib (replace nb with the problem number)
% prefer books, research articles, or internet sources that are likely to remain available over time
% as much as possible offer several options, including at least one which provide a detailed study of the problem
% if available include links to programs/code solving the problem
% wikipedia is NOT acceptable as a unique reference
\singlespacing
\printbibliography[title={References.},resetnumbers=true,heading=subbibliography]

\end{document}
