\documentclass[catalog.tex]{subfiles}

% do not write anything in the preamble

\begin{document}

\def\pbname{Traveling salesman problem} %change this, do not use any number, just the name

\section{\pbname} 

% only for overview, so short description (no more than 1-2 lines)
\begin{overview}
\item [Algorithm:] Christofides Algorithm~(algo.~\ref{alg:\currfilebase}) 
	% -	must match the label of the algorithm 
	% - when writing more than one algo use alg:\currfilebase_a, alg:\currfilebase_b, etc.
\item [Input:] a universe of cities $V = \{1, 2, ..., n\}$, a metric distance function $d: V \times V \to R^+$.
\item [Complexity:] $\mathcal{O}(n^2 \log n)$, for approximate algorithm, actually we care more about the accuracy.
\item [Data structure compatibility:] graph and matrix
\item [Common applications:] supply chain management and chip production
\end{overview}


\begin{problem}{\pbname}
	We assume a universe of cities $V = \{1, 2, ..., n\}$, the metric distance function is $d: V \times V \to R^+$. We want to find a path(a traverse of the cities) $k(1), k(2), ..., k(n)$, such that the total cost $\sum_{i=1}^{n-1}d_{k(i)k(i+1)} + d_{k(n)k(1)}$ is minimized.
\end{problem}

\subsection*{Description}
For a simple introduction to approximate algorithm, please refer to problem 71, or read the book above, or take course VG441, which is a course offered by JI. Unlike humanity course like VR477 introduction to humanity project, VG441 is a real CS course.\newline

TSP, shorthand for traveling salesman problem originates from a salesman that wants to travel through every city exactly once using shortest distance. It is a classic problem in the field of operations research and computer science. It is related to problems like traveling purchaser problem. We would like to research on TSP because it has tons of applications in real life. For example supply chain management has to take TSP into account so that the goods are delivered in a fast and cheap way. TSP can also be modified to cope with problems in chip production and DNA measurement. If you are interested in TSP and wants to study further, you may refer to a book(collection of papers TAT) called \emph{the Design of Approximate Algorithms}\cite{williamson2011design}. You will find variation like asymmetric TSP and euclidean TSP.
\newline

Detailed definition of this problem is shown below. Assume we have a universe of cities $V = \{1, 2, ..., n\}$, and the symmetric distance between every city pair is defined as $C = (c_{ij})$. We assume a distance function to be $d: V \times V \to R^+$. The distance can have meanings like physical distance, travel expenses, travel time cost and so on. Notice that the cost should be nonnegative. Since there does not exist an approximation algorithm if we do not restrict the cost to be metric\cite{vg441}. So we should specify the cost to be metric. here metric means the cost be symmetry and it should satisfy the triangular inequality. Given the cities and the cost function between them, we would like to find a path with minimum cost that visits every city in the universe exactly once and then return to the starting point. \newline

Time complexity analysis. For the MST, we can use the prim's algorithm taught in VE477 and the time complexity should be $\mathcal{O}(E + logV) = \mathcal{O} (n^2)$ in our case\cite{ve477}. Also $\mathcal{O}(n^2)$ is enough for find the set of odd degree vertices. After that we do the minimum cost matching which is of complexity $\mathcal{O}(nm \log n)$, in our case, m is $\mathcal{O}(n)$. Add edges is linearly cheap, find a walk is also linearly cheap and we can ignore it. Shortcutting in the last step in the algorithm needs only linear time as well. In short, the time complexity of this Christofides algorithm depends on the time complexity of finding minimum cost matching.

% add comment in the pseudocode: \cmt{comment}
% define a function name: \SetKwFunction{shortname}{Name of the function}
% use the defined function: \shortname{$variables$}
% use the keyword ``function'': \Fn{function name}, e.g. \Fn{\shortname{$var$}}
\begin{Algorithm}[Christofides\label{alg:\currfilebase}]
	% -	must match the reference in the overview
	% - when writing more than one algo use alg:\currfilebase_a, alg:\currfilebase_b, etc.
	%\SetKwFunction{myfunction}{MyFunction}	
	\Input{a universe of cities $V = \{1, 2, ..., n\}$, a metric distance function $d: V \times V \to R^+$.}
	\Output{a path that visits every city in the universe exactly once and then return to the starting point}
	%	\Fn{\myfunction{$a,b$}}{
	%	}
	\BlankLine
	Compute the minimum spanning tree(MST) M of (V,d)

	Compute the minimum cost matching K on the vertices of M with odd degree

	Add edges of K to M and we get an Eulerian graph D'

	find a walk W' that travels through each edge of D' eactly once

	shortcut W' by skipping revisited vertices to get a TSP path T'

	\Ret T'

\end{Algorithm}


% include references where to find information on the given problem using latex bibliography
% insert references in the text (\cite{}) and write bibliography file in problem-nb.bib (replace nb with the problem number)
% prefer books, research articles, or internet sources that are likely to remain available over time
% as much as possible offer several options, including at least one which provide a detailed study of the problem
% if available include links to programs/code solving the problem
% wikipedia is NOT acceptable as a unique reference
\singlespacing
\printbibliography[title={References.},resetnumbers=true,heading=subbibliography]

\end{document}
